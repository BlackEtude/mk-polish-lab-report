\documentclass[11pt]{mk-polish-lab-report}

\usepackage{lipsum}

\university{Politechnika Wrocławska}
\major{Informatyka, inż. I st.}
\tutor{dr hab. Paweł Zieliński}
\coursegroup{czwartek TN, 11:15}

\author{Genowefa Bolesna-Mordęga}
\studentnumber{12345}
\title{Obliczenia Naukowe}
\topic{Lista 1}


%% 
\geometry{top=2.5cm,bottom=2cm,left=2.5cm,right=2.5cm}



\begin{document}

\maketitle

\section{Rozpoznanie arytmetyki}
\lipsum[1-2]

	\begin{table}[!h]
        \centering
        \footnotesize
        \sisetup{
        table-number-alignment = right,
		table-figures-integer  = 10,
		table-figures-decimal = 16,
		table-figures-exponent=2
		}
		\begin{tabular}{lSS}
		\toprule
			{podpunkt} & {\texttt{Float32}} & {\texttt{Float64}} \\ \midrule
			$a$ & -111110.4999443 & 1.0251881368296672e-10 \\ 
 			$b$ & -0.4543457 & -1.5643308870494366e-10 \\
 			$c$ & -0.5 & 0.0e0 \\
 			$d$ & -0.5 & 0.0e0 \\\bottomrule
 		\end{tabular}
 		\caption{Obliczanie iloczynu skalarnego wektorów}
		\label{table:8}
	\end{table}	


\lipsum[3-5]

\end{document}
